\section{Motivating examples}
\dcproblogsty is a probabilistic logic programming language derived from two related but separate probabilistic logic programming languages,  \problogsty and Distributional Clauses (DC).
It inherits its Prolog syntax and the concept of probabilistic facts from \problogsty, and combines this with the ideas to define and query continuous random variables in DC, using DC's infix 
\probloginline{~}\lstinline[columns=fixed]|/2| predicate to define  {\em distributional facts}. Distributional facts allow a user to express random variables that are distributed according to a certain probability distribution specified by a random function symbol.

In the following example, the distributional fact in line 3 introduces the random variable \verb|temperature| and associates it with the distribution term \verb|normal(20,5)|. This random variable is then queried in the clause in line 8 using comparison predicate \verb|</2|, again written in infix notation. 
\begin{example}\label{example:dcproblog_machine}
	In this example we model the temperature as a continuous random variable distributed according to a normal distribution with mean $20$ and standard deviation $5$ (Line \ref{program:dcproblog_machines_temp}).
	\begin{problog*}{linenos}
machine(1). machine(2).

temperature ~ normal(20,5). @\label{program:dcproblog_machines_temp}@
0.99::cooling(1).
0.95::cooling(2).

works(N):- machine(N), cooling(N).
works(N):- machine(N), temperature<25.0. @\label{program:dcproblog_machines_work_temp}@

evidence(works(2)).  @\label{program:dcproblog_machines_evidence}@
query(works(1)).
	\end{problog*}
\end{example}


Distributional facts are not limited to describe continuous random variables but can, for example, also model discrete probability distributions, of which an example would be the Poisson distribution.
\begin{example}\label{example:dcproblog_poisson_population}
	In this example we show how to model the size of a group of people as a Poisson distribution and answer queries about the size of the group using the comparison predicates  \probloginline{>}\lstinline[columns=fixed]{/2} and \probloginline{=:=}\lstinline[{columns=fixed}]{/2}.
	\begin{problog*}{linenos}
n_people ~ poisson(6).

more_than_five:- n_people>5@.@
exactly_five:- n_people=:=5@.@

query(more_than_five).
query(exactly_five).
	\end{problog*}
	
\end{example}

We will formally introduce \dcproblogsty in Section~\ref{}. To introduce the semantics, in the next Section, we start by introducing DC-PLP, an abstract version of the language that omits syntactic sugar. 



